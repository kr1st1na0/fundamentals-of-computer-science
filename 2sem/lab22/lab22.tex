\documentclass[a4paper]{report} %
\usepackage[utf8]{inputenc}
\usepackage[russian]{babel} %
\usepackage[paperheight=29.7cm,paperwidth=21cm,textwidth=19cm]{geometry}
\usepackage{amsmath}
\usepackage{graphicx}
\usepackage[colorlinks=true, allcolors=blue]{hyperref}
\usepackage{titling}
\usepackage{amsmath}

\thispagestyle{empty}

\begin{document}

\noindent
\textbf{\Large Отчёт по лабораторной работе №22 по курсу "Языки и методы программирования"} \\

\hline
 
 \textbf{\\Студент группы:~} \underline{М8О-108Б-22 Былькова Кристина Алексеевна, № по списку 2} \\
\textbf{Контакты e-mail:} \href{mailto:kristina.bilckova@yandex.ru}{kristina.bilckova@yandex.ru} \\
\textbf{Работа выполнена} "25" \underline{марта 2023} г. \\
\textbf{Преподаватель:} \underline{асп. каф. 806 Сахарин Никита Александрович} \\
\textbf{Входной контроль знаний с оценкой:} \underline{-}\\
\textbf{Отчёт сдан} "25" \underline{марта 2023} г., \textbf{итоговая оценка} \underline{-} \\
\textbf{Подпись преподавателя:} \rule{60}{0.03}


\section*{1. Тема}
Издательская система TeX
\blindtext

\section*{2. Цель работы}
Ознакомиться с системой TeX.    
\blindtext

\section*{3. Задание}
Вёрстка отчёта по лабораторной работе №22 в LaTeX.
\blindtext

\section*{4. Оборудование}
\begin{itemize}
    \item Процессор: AMD Ryzen9-5900HS, 8 ядер
    \item ОП: 16gb
    \item SSD: 1 Tb SSD
    \item Монитор: 15.6" - 2560x1440
    \item Графика: NV GeForce RTX 3080
\end{itemize}

\section*{5. Программное обеспечение}
\begin{itemize}
    \item Операционная система семейства: VirtualBox 6.1.38 - Ubuntu 22.04.01 LTS
    \item Интерпретатор команд: bash версия 4.4.19
    \item Система программирования: -
    \item Редактор текстов: -
    \item Утилиты операционной системы: -
    \item Прикладные системы и программы: -
    \item Местонахождение и имена файлов программ и данных на домашнем компьютере: /home/kristina
\end{itemize}

\section*{6. Идея, метод, алгоритм решения задачи (в формах: словесной, псевдокода, графической [блок-схема, диаграмма, рисунок, таблица] или формальные спецификации с пред- и постусловиями)}
\paragraph{}
TEX – система компьютерной верстки, разработанная американским профессором информатики Дональдом Кнутом в целях создания компьютерной
типографии. В нее входят средства для секционирования документов, для
работы с перекрестными ссылками.
В отличие от обыкновенных текстовых процессоров и систем компьютерной верстки, построенных по принципу WYSIWYG – What You See Is
What You Get («что видишь, то и получишь»), в TEX’е пользователь лишь
задает текст и его структуру, а TEX самостоятельно на основе выбранного
пользователем шаблона форматирует документ, заменяя при этом дизайнера
и верстальщика.
\par
Документы набираются на собственном языке разметки в виде обычных
ASCII-файлов, содержащих информацию о форматировании текста или выводе изображений. Эти файлы (обычно имеющие расширение «.tex») транслируются специальной программой в файлы «.dvi» (device independent – «независимые от устройства»), которые могут быть отображены на экране или напечатаны. DVI-файлы можно специальными программами преобразовать в PostScript, PDF или другой электронный формат.
\par
Tex является первой системой, в которой угадано основное направление развития представлений текстовой информации в распределенных информационных системах. Используемые в TeX'e, и особенно в его клонах, принципы разметки текста и отделения содержания от представления, лежат в основе современных подходов к обработке структурированной текстовой информации. Конечно, в ТеХ'е все эти возможности используются нерегулярно, и в незначительных масштабах. Кроме того, формализм ТеХ'а не соответствует современным формализмам разметки текстов, основанных на языке XML.

\section*{7. Сценарий выполнения работы [план работы, первоначальный текст программы в черновике (можно на отдельном листе) и тесты либо соображения по тестированию]}

1. Чтение литературы по соответствующей теме  \\
2. Изучение примеров, находящихся в открытом доступе \\
3. Вёрстка отчёта через Online LaTeX Editor Overleaf\\

Пункты 1-7 отчета составляются сторого до начала лабораторной работы. Допущен к выполнению работы.
Подпись преподавателя: \rule{60}{0.03}

\thispagestyle{empty}

\section*{8. Распечатка протокола}

В ходе выполнения данной лабораторной работы были изучены основы работы в системе LaTeX: оформление таблиц, математических формул и всего документа в целом.
\\
\par
\sqrt[n]{1+x+x^2+x^3+\dots+x^n}
\\
\par
\int sin(x)\, \mathrm{d}x
\\
\par
A_{m,n} = 
 \begin{pmatrix}
  a_{11} & a_{12} & \cdots & a_{1n} \\
  a_{21} & a_{22} & \cdots & a_{2n} \\
  \vdots  & \vdots  & \ddots & \vdots  \\
  a_{m1} & a_{m2} & \cdots & a_{mn} 
 \end{pmatrix}

\section*{9. Дневник отладки должен содержать дату и время сеансов отладки и основные события (ошибки в сценарии и программе, нестандартные ситуации) и краткие комментарии к ним. В дневнике отладки приводятся сведения об использовании других ЭВМ, существенном участии преподавателя и других лиц в написании и отладке программы.}

\begin{tabular}{ | c | c | c | c | c | c | c | }
\hline
№ & Лаб. или дом. & Дата & Время & Событие & Действие по исправлению & Примечание \\
\hline
1 & дом. & 25.03.23 & 13:00 & Выполнение лабораторной работы & - & - \\
\hline
\end{tabular}

\section*{10. Замечания автора по существу работы}
-
\section*{11. Выводы}

Было выяснено, что среди множества издательских систем, используемых для подготовки публикаций, особое местро занимает система TeX, широко применяющаяся в научных кругах при подготовке научных статей, докладов, презентаций, монографий, тезисов и т.д. В результате выполнения работы, были приобретены навыки работы , которые будут полезны для выполнения других лабораторных работ и курсовых проектов.
\\
\\
Недочёты при выполнении задания могут быть устранены следующим образом: —
\\
Подпись студента: \rule{60}{0.03}

\thispagestyle{empty}

\end{document}